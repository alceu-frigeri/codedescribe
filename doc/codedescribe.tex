%%%==============================================================================
% WinEdt pragmas
% !Mode:: "TeX:EN"
% Default Compile engines:
% !TEX program = pdflatex
% !PDFTeXify ext =  --enable-etex  --restrict-write18
% !PDFLaTeX ext  =  --enable-etex  --restrict-write18
% !BIB program = none
%%%==============================================================================
%% Copyright 2023-present by Alceu Frigeri
%%
%% This work may be distributed and/or modified under the conditions of
%%
%% * The [LaTeX Project Public License](http://www.latex-project.org/lppl.txt),
%%   version 1.3c (or later), and/or
%% * The [GNU Affero General Public License](https://www.gnu.org/licenses/agpl-3.0.html),
%%   version 3 (or later)
%%
%% This work has the LPPL maintenance status *maintained*.
%%
%% The Current Maintainer of this work is Alceu Frigeri
%%
%% This is version {1.7} {2025/04/21}
%%
%% The list of files that compose this work can be found in the README.md file at
%% https://ctan.org/pkg/codedescribe
%%
%%%==============================================================================
%\documentclass[dctools,english]{ufrgscca}
\documentclass{article}
\RequirePackage[verbose,a4paper,marginparwidth=28mm,top=3cm,bottom=1.5cm,hmargin={45mm,25mm},marginparsep=2.5mm,columnsep=10mm,asymmetric]{geometry}
\usepackage{codedescribe}
\usepackage{enumitem}

\RequirePackage[hidelinks ,hypertexnames=false]{hyperref}
\begin{document}

%%% TO BE DETERMINED !! \parindent
%%%\setlength\parindent{0pt}

\tstitle{
  author={Alceu Frigeri\footnote{\tsverb{https://github.com/alceu-frigeri/codedescribe}}},
  date={\tsdate},
  title={The codedescribe and codelisting Packages\break Version \PkgInfo{codedescribe}{version}}
  }
  
\begin{typesetabstract}

This documentation package is designed to be `as class independent as possible', depending only on \tsobj[pkg]{expl3,scontents,listing,pifont}. Unlike other packages of the kind, a minimal set of macros/commands/environments is defined: most/all defined commands have an 'object type' as a \tsobj[key]{keyval} parameter, allowing for an easy expansion mechanism (instead of the usual 'one set of macros/environments' for each object type).

No assumption about page layout is made (besides `having a marginpar'),  or underlying macros, so that it can be used with any document class.

\end{typesetabstract}

\tableofcontents

\section{Introduction}


This package aims to document both \tsobj[code]{Document} level (i.e. final user) commands, as well \tsobj[code]{Package/Class} level commands. It's fully implemented using \tsobj[pkg]{expl3} syntax and structures, in special \tsobj[pkg]{l3coffins, l3seq, l3keys}. Besides those \tsobj[pkg]{scontents,listing} packages are used to typeset code snippets. The package \tsobj[pkg]{pifont} is needed just to typeset those (open)stars, in case one wants to mark a command as (restricted) expandable.
 
 
No other package/class is needed, any class can be used as the base one, which allows to demonstrate the documented commands with any desired layout.

\tsobj[pkg]{codelisting} defines a few macros to display and demonstrate \LaTeX~ code (using \tsobj[pkg]{listings} and \tsobj[pkg]{scontents}), whilst \tsobj[pkg]{codedescribe} defines a series of macros to display/enumerate macros and environments (somewhat resembling the \tsobj[pkg]{doc3} style).

\subsection{Single versus Multi-column Classes}
This package 'can' be used with multi-column classes, given that the \tsobj[code]{\linewidth,\columnsep} are defined appropriately. \tsobj{\linewidth} shall defaults to text/column real width, whilst \tsobj{\columnsep}, if needed (2 or more columns) shall be greater than \tsobj{\marginparwidth} plus \tsobj{\marginparsep}.

\subsection{Current Version}
This doc regards to \tsobj[pkg]{codedescribe} version \PkgInfo{codedescribe}{version}~ and \tsobj[pkg]{codelisting} version \PkgInfo{codelisting}{version}. Those two packages are fairly stable, and given the \tsobj[meta]{obj-type} mechanism (see below, \ref{obj-type-def}) they can be easily extended without changing it's interface.

\section{codelisting Package}

It requires two packages: \tsobj[pkg]{listings,scontents},  defines an environment: \tsobj[env]{codestore} and 
 3 main commands: \tsobj[code]{\tscode,\tsdemo,\tsresult} and 1 auxiliary command: \tsobj{\setcodekeys}.

\subsection{In Memory Code Storage}
Thanks to \tsobj[pkg]{scontents} (\tsobj[pkg]{expl3} based) it's possible to store \LaTeX~ code snippets in a \tsobj[pkg]{expl3} key.


\begin{codedescribe}[env,rulecolor=white]{codestore}
	\begin{codesyntax}
		\tsmacro{\begin{codestore}}[stcontents-keys]{}
    \tsmacro{\end{codestore}}{}
	\end{codesyntax}
This environment is an alias to \tsobj[env]{scontents} environment (from \tsobj[pkg]{scontents} package), all \tsobj[pkg]{scontents} keys are valid, with two additional ones: \tsobj[key]{st,store-at} which are aliases to the \tsobj[key]{store-env} key. If an 'isolated' \tsobj[key,meta]{st-name} is given (unknown \tsobj[key]{key}), it is assumed 'by Default' that the environment body shall be stored in it (for use with \tsobj[code]{\tscode,\tsdemo}).
\end{codedescribe}



\subsection{Code Display/Demo}\label{codelist}

\begin{codedescribe}{\setcodekeys}
	\begin{codesyntax}%
		\tsmacro{\setcodekeys}{code-keys}
	\end{codesyntax}

One has the option to set \tsmeta{code-keys} (see \ref{code-keys}) per \tsmacro{\tscode}{}/\tsmacro{\tsdemo}{} call, or \emph{globally}, better said, \emph{in the called context group} .\\
\begin{tsremark}[N.B.:]
All \tsobj[code]{\tscode,\tsdemo} commands create a local group  in which the \tsmeta{code-keys} are defined, and discarded once said local group is closed. \tsmacro{\setcodekeys}{} defines those keys in the \emph{current} context/group.
\end{tsremark}
\end{codedescribe}

\begin{codedescribe}[code,update=2024/01/06]{\tscode*,\tsdemo*,\tsresult*}
	\begin{codesyntax}%
		\tsmacro{\tscode*}[code-keys]{st-name}
		\tsmacro{\tsdemo*}[code-keys]{st-name}
		\tsmacro{\tsresult*}[code-keys]{st-name}
	\end{codesyntax}
\tsmacro{\tscode}{} just typesets \tsobj[meta]{st-name} (the key-name created with \tsobj[env]{stcode}), in verbatim mode with syntax highlight. The non-star version centers it and use just half of the base line. The star version uses the full text width.

\tsmacro{\tsdemo*}{} first typesets \tsobj[meta]{st-name}, as above, then it \emph{executes} said code. The non-start versions place them side-by-side, whilst the star versions places one following the other.

(new 2024/01/06) \tsmacro{\tsresult*}{} only \emph{executes} said code. The non-start versions centers it and use just half of the base line, whilst the star versions uses the full text width.

\end{codedescribe}


For Example:

\begin{codestore}[st=democodestore]
\begin{codestore}[stmeta]
		Some \LaTeX~coding, for example: \ldots.
\end{codestore}  
This will just typesets \tsobj[key]{stmeta}:
\tscode*[codeprefix={Sample Code:}] {stmeta}
and this will demonstrate it, side by side with source code:
\tsdemo[numbers=left,ruleht=0.5, 
    codeprefix={inner sample code},
    resultprefix={inner sample result}] {stmeta}
\end{codestore}
\tsdemo*[emph2={tscode,tsdemo},emph={stmeta},keywd2={codestore}]{democodestore}



\subsubsection{Code Keys}\label{code-keys}
Using a \tsobj[key]{key}\,=\tsobj[value]{value} syntax, one can fine tune \tsobj[pkg]{listings} syntax highlight.

\begin{codedescribe}[key]{settexcs,texcs,texcsstyle}
\begin{codesyntax}% Also
	\tsobj[key]{settexcs,settexcs2,settexcs3}
	\tsobj[key]{texcs,texcs2,texcs3}
	\tsobj[key]{texcsstyle,texcs2style,texcs3style}
\end{codesyntax}
Those define sets of \LaTeX~commands (csnames), the \tsobj[key]{set} variants initialize/redefine those sets (an empty list, clears the set), while the others extend those sets. The \tsobj[key]{style} ones redefines the command display style (an empty \tsobj[meta]{value} resets the style to it's default).\\
\end{codedescribe}


\begin{codedescribe}[key]{setkeywd,keywd,keywdstyle}
\begin{codesyntax}	%
	\tsobj[key]{setkeywd,setkeywd2,setkeywd3}
	\tsobj[key]{keywd,keywd2,keywd3}
	\tsobj[key]{keywdstyle,keywd2style,keywd3style}
\end{codesyntax}
Same for other \emph{keywords} sets.\\
\end{codedescribe}

\begin{codedescribe}[key]{setemph,emph,emphstyle}
\begin{codesyntax}	%
	\tsobj[key]{setemph,setemph2,setemph3}
	\tsobj[key]{emph,emph2,emph3}
	\tsobj[key]{emphstyle,emph2style,emph3style}
\end{codesyntax}
for some extra emphasis sets.\\
\end{codedescribe}

\begin{codedescribe}[key]{numbers,numberstyle}
\begin{codesyntax} %
	\tsobj[key]{numbers,numberstyle}
\end{codesyntax}
\tsobj[key]{numbers} possible values are \tsobj[value]{none} (default) and \tsobj[value]{left} (to add tinny numbers to the left of the listing). With \tsobj[key]{numberstyle} one can redefine the numbering style.\\
\end{codedescribe}

\begin{codedescribe}[key]{stringstyle,codestyle}
\begin{codesyntax} %
	\tsobj[key]{stringstyle,commentstyle}
\end{codesyntax}
to redefine \tsobj[key]{strings} and \tsobj[key]{comments} formatting style.\\
\end{codedescribe}

\begin{codedescribe}[key]{bckgndcolor}
\begin{codesyntax} %
	\tsobj[key]{bckgndcolor}
\end{codesyntax}
to change the listing background's color.\\
\end{codedescribe}

\begin{codedescribe}[key]{codeprefix,resultprefix}
\begin{codesyntax} %
	\tsobj[key]{codeprefix,resultprefix}
\end{codesyntax}
those set the \tsobj[key]{codeprefix} (default: \LaTeX~Code:) and \tsobj[key]{resultprefix} (default: \LaTeX~Result:)
\end{codedescribe}

\begin{codedescribe}[key]{parindent}
\begin{codesyntax} %
	\tsobj[key]{parindent}
\end{codesyntax}
Sets the indentation to be used when 'demonstrating' \LaTeX\ code (\tsobj[code]{\tsdemo}). Defaults to whatever value \tsobj[code]{\parindent} was when the package was first loaded.
\end{codedescribe}

\begin{codedescribe}[key]{ruleht}%-very-long-long-long,xyz,xyh,kjh,ljd,kls}
\begin{codesyntax} %
	\tsobj[key]{ruleht}
\end{codesyntax}
When typesetting the 'code demo' (\tsobj{\tsdemo}) a set of rules is drawn. The Default, 1, equals to \tsobj{\arrayrulewidth} (usually 0.4pt).
\end{codedescribe}

\begin{codedescribe}[key,new=2023/11/18]{basicstyle}%-very-long-long-long,xyz,xyh,kjh,ljd,kls}
\begin{codesyntax} %
	\tsobj[key]{basicstyle}
\end{codesyntax}
Sets the base font style used when typesetting the 'code demo', default being \tsobj{\footnotesize\ttfamily}
\end{codedescribe}


\section{codedescribe Package}

This package aims at minimizing the number of commands, having the object kind (if a macro, or a function, or environment, or variable, or key ...) as a parameter, allowing for a simple 'extension mechanism': other object types can be easily introduced without having to change, or add commands.

\subsection{Package Options}
It has a single package option:

\begin{describelist}{option}
\describe{nolisting}{it will suppress the \tsobj[pkg]{codelisting} package load. In case it isn't needed or another listing package will be used.}
\end{describelist}

\subsection{Object Type keys}\label{obj-type-def}

The applied text format is defined in terms of \tsobj[arg]{obj-types}, which are defined in terms of \tsobj[arg]{format-groups} and each one defines a 'formatting function', 'font shape', bracketing, etc. to be applied.

\subsubsection{Format Keys}\label{format-keys}
There is a set of primitive \tsobj[arg]{format-keys} used to define  \tsobj[arg]{format-groups,obj-types}, which are:

\begin{describelist*}{keys}
  \describe {meta} { to typeset between angles,}
  \describe {xmeta} { to typeset *verbatim* between angles,}
  \describe {verb} { to typeset *verbatim*,}
  \describe {xverb} { to typeset *verbatim*, suppressing all spaces,}
  \describe {code} { to typeset *verbatim*, suppressing all spaces and replacing a TF by \underline{TF},}
  \describe {nofmt} { in case of a redefinition, to remove the 'base' formatting,}
  \describe {slshape} { to use a slanted font shape,}
  \describe {itshape} { to use an italic font shape,}
  \describe {noshape} { in case of a redefinition, to remove the 'base' shape,}
  \describe {lbracket}  { defines the left bracket (when using \tsobj{\tsargs}). \textbf{Note:} this key must have an associated value,}
  \describe {rbracket}  { defines the right bracket (when using \tsobj{\tsargs}). \textbf{Note:} this key must have an associated value,}
  \describe {color}   { defines the text color. \textbf{Note:} this key must have an associated value (a color, as understood by \tsobj[pkg]{xcolor} package).}
\end{describelist*}

\subsubsection{Format Groups}
Using \tsobj{\defgroupfmt} one can (re-)define custom \tsobj[arg]{format-groups}. There is, though, a set of pre-defined  ones as follow:

\begin{describelist*}{keys}
  \describe {meta} {which sets \tsobj[keys]{meta,color}}
  \describe {verb} {which sets \tsobj[keys]{color}}
  \describe {oarg} {which sets \tsobj[keys]{meta,color}}
  \describe {code} {which sets \tsobj[keys]{code,color}}
  \describe {syntax} {which sets \tsobj[keys]{color}}
  \describe {keyval} {which sets \tsobj[keys]{slshape,color}}
  \describe {option} {which sets \tsobj[keys]{color}}
  \describe {defaultval} {which sets \tsobj[keys]{color}}
  \describe {env}  {which sets \tsobj[keys]{slshape,color}}
  \describe {pkg}  {which sets \tsobj[keys]{slshape,color}}
\end{describelist*}
\begin{tsremark}
\tsobj[keys]{color} was used in the list above just as a 'reminder' that a color is defined/associated with the given group.
\end{tsremark}

\subsubsection{Object Types}\label{obj-types}
Using \tsobj{\defobjectfmt} one can (re-)define custom \tsobj[arg]{obj-types}.
Similarly, there is a set of predefined ones, as follow:

\begin{describelist*}{keys}
  \describe {arg, meta} {based on (group) \tsobj[key]{meta}}
  \describe {verb, xverb} {based on (group) \tsobj[key]{verb} plus \tsobj[key,sep=or]{verb,xverb}}
  \describe {marg} {based on (group) \tsobj[key]{meta} plus brackets}
  \describe {oarg, parg, xarg} {based on (group) \tsobj[key]{oarg} plus brackets}
  \describe {code, macro, function} {based on (group) \tsobj[key]{code}}
  \describe {syntax} {based on (group) \tsobj[key]{syntax}}
  \describe {keyval, key, keys, values} {based on (group) \tsobj[key]{keyval}}
  \describe {option} {based on (group) \tsobj[key]{option}}
  \describe {defaultval} {based on (group) \tsobj[key]{defaultval}}
  \describe {env}  {based on (group) \tsobj[key]{env}}
  \describe {pkg, pack}  {based on (group) \tsobj[key]{pkg}}
\end{describelist*}

\subsubsection{Customization}\label{format-custom}

To create  user defined groups/objects or change the pre-defined ones:


\begin{codedescribe}[code,new=2023/05/16]{\defgroupfmt}
\begin{codesyntax} %
	\tsmacro{\defgroupfmt}{format-group,format-keys}
\end{codesyntax}
\tsobj[marg]{format-group} is the name of the new group (or one being redefined, which can be one of the standard ones). \tsobj[marg]{format-keys} is any combination of the keys defined in \ref{format-keys}
\end{codedescribe}

For example, one can redefine the \tsobj[key]{code group} standard color with \tsobj{\defgroupfmt{code}{color=red}} and all \tsobj[key]{obj-types} based on it will be typeset in red (in the standard case: \tsobj[key]{code, macro, function} objects).


\begin{codedescribe}[code,new=2023/05/16]{\defobjectfmt}
\begin{codesyntax} %
	\tsmacro{\defobjectfmt}{obj-type,format-group,format-keys}
\end{codesyntax}
\tsobj[marg]{obj-type} is the name of the new \tsobj[arg]{object} being defined (or redefined),
\tsobj[marg]{format-group} is the base group to be used. \tsobj[marg]{format-keys} allows for further differentiation.
\end{codedescribe}
For instance, the many optional \tsobj[arg]{*arg} are defined as follow:
\begin{codestore}[demo.fmtdef]
\colorlet {c__codedesc_oarg_color} { gray!90!black }

\defgroupfmt  {oarg} { meta , color=c__codedesc_oarg_color }

\defobjectfmt {oarg} {oarg} { lbracket={[} , rbracket={]} }
\defobjectfmt {parg} {oarg} { lbracket={(} , rbracket={)} }
\defobjectfmt {xarg} {oarg} { lbracket={<} , rbracket={>} }
\end{codestore}
\tscode*[codeprefix=~,keywd={oarg,arg,parg,xarg},keywd2={meta,color},emph={lbracket,rbracket}]{demo.fmtdef}

\subsection{Environments}

\begin{codedescribe}[env,new=2023/05/01,update=2023/05/01,note={this is an example},update=2024/02/16]{codedescribe}
\begin{codesyntax}
\tsmacro{\begin{codedescribe}}[obj-type]{csv-list}
\ldots
\tsmacro{\end{codedescribe}}{}
\end{codesyntax}
This is the main environment to describe \tsobj[env]{Macros, Functions, Variables, Environments, etc.}  \tsobj[meta]{csv-list} is typeset in the margin. The optional \tsobj[oarg]{obj-type} (see \ref{obj-type-def} and \ref{obj-types}) defines the object-type format. 
\end{codedescribe}
\begin{tsremark}[Note~1:]
One can change the rule color with the key \tsobj[keys]{rulecolor}, for instance \tsmacro{\begin{codedescribe}[rulecolor=white]}{} will remove the rules.
\end{tsremark}
\begin{tsremark}[Note~2:]
Besides that, one can use the keys \tsobj[keys]{new,update,note} to further customize it. (2024/02/16 these keys can also be used multiple times).
\end{tsremark}
\begin{tsremark}[Note~3:]
Finally, one can use \tsobj[key]{EXP,rEXP} to add a star \ding{72} or a hollow star \ding{73}, as per expl3/doc3 conventions (if expandable, restricted expandable or not).
\end{tsremark}

\begin{codedescribe}[env]{codesyntax}
\begin{codesyntax}
\tsmacro{\begin{codesyntax}}{}
\ldots
\tsmacro{\end{codesyntax}}{}
\end{codesyntax}
The \tsobj[env]{codesyntax} environment sets the fontsize and activates \tsmacro{\obeylines,\obeyspaces}{}, so one can list macros/cmds/keys use, one per line.
\end{codedescribe}

\begin{tsremark*}
\tsobj[env]{codesyntax} environment shall appear only once, inside of a \tsobj[env]{codedescribe} environment. Take note, as well, this is not a verbatim environment!
\end{tsremark*}

For example, the code for  \tsobj[env]{codedescribe} (entry above) is:

\begin{codestore}[demoD]
\begin{codedescribe}[env,new=2023/05/01,update=2023/05/01,note={this is an example},update=2024/02/16]{codedescribe}
  \begin{codesyntax}
    \tsmacro{\begin{codedescribe}}[obj-type]{csv-list}
    \ldots
    \tsmacro{\end{codedescribe}}{}
  \end{codesyntax}
  This is the main ...
\end{codedescribe}
\end{codestore}  

\tscode*{demoD}



\begin{codedescribe}[env]{describelist,describelist*}
  \begin{codesyntax}
\tsmacro{\begin{describelist}}[item-indent]{obj-type}
..\tsmacro{\describe}{item-name,item-description}
..\tsmacro{\describe}{item-name,item-description}
\ldots
\tsmacro{\end{describelist}}{}
  \end{codesyntax}
This sets a \tsobj[env]{description} like 'list'. In the non-star version the \tsobj[marg]{items-name} will be typeset on the marginpar. In the star version, \tsobj[marg]{item-description} will be indented by \tsobj[oarg]{item-indent} (defaults to: 20mm).
\tsobj[marg]{obj-type} defines the object-type format used to typeset \tsobj[marg]{item-name}. 
\end{codedescribe}

\begin{codedescribe}[code]{\describe}
\begin{codesyntax}
\tsmacro{\describe}{item-name,item-description}
\end{codesyntax}
This is the \tsobj[env]{describelist} companion macro. In case of the \tsobj[env]{describe*}, \tsobj[marg]{item-name} will be typeset in a box \tsobj[oarg]{item-ident} wide, so that \tsobj[marg]{item-description} will be fully indented, otherwise \tsobj[marg]{item-name} will be typed in the marginpar.
\end{codedescribe}


\subsection{Commands}


\begin{codedescribe}[code]{\typesetobj,\tsobj}
\begin{codesyntax}
\tsmacro{\typesetobj}[obj-type]{csv-list}
\tsmacro{\tsobj}[obj-type]{csv-list}
\end{codesyntax}
This is the main typesetting command (most of the others are based on this). It can be used to typeset a single 'object' or a list thereof. In the case of a list, each term will be separated by commas. The last two  by \tsobj[key]{sep} (defaults to: and).
\end{codedescribe}
\begin{tsremark}
The last 'separator' can be changed with the key \tsobj[keys]{sep}, for instance \tsverb[code]{\tsobj[env,sep=or] {}} (it will produce an 'or' list of environments). Additionally,  the key \tsobj[keys]{comma} will change the last separator to a single comma, as in \tsverb[code]{\tsobj[env,comma] {}}.
\end{tsremark}

\begin{codedescribe}[code]{\typesetargs,\tsargs}
\begin{codesyntax}
\tsmacro{\typesetargs}[obj-type]{csv-list}
\tsmacro{\tsargs}[obj-type]{csv-list}
\end{codesyntax}
 Those will typeset \tsobj[marg]{csv-list} as a list of parameters, like \tsargs[oarg]{arg1,arg2,arg3}, or \tsargs[marg]{arg1,arg2,arg3}, etc. \tsobj[oarg]{obj-type} defines the formating AND kind of brackets used (see \ref{obj-type-def}): \tsverb{[]} for optional arguments (oarg), \tsverb{{}} for mandatory arguments (marg), and so on.
\end{codedescribe}


\begin{codedescribe}[code]{\typesetmacro,\tsmacro}
\begin{codesyntax}
\tsmacro{\typesetmacro}{macro-list}\tsargs[oarg]{oargs-list}\tsargs[marg]{margs-list}
\tsmacro{\tsmacro}{macro-list}\tsargs[oarg]{oargs-list}\tsargs[marg]{margs-list}
\end{codesyntax}
This is just a short-cut for\par \tsobj[code]{\tsobj [code] {macro-list}} \tsobj[code]{\tsargs [oarg] {oargs-list}} \tsobj[code]{\tsargs [marg] {margs-list}}.
\end{codedescribe}

\begin{codedescribe}[code]{\typesetmeta,\tsmeta}
\begin{codesyntax}
\tsmacro{\typesetmeta}{name}
\tsmacro{\tsmeta}{name}
\end{codesyntax}
 Those will just typeset \tsobj[meta]{name} between left/right 'angles' (no other formatting).
\end{codedescribe}

\begin{codedescribe}[code]{\typesetverb,\tsverb}
\begin{codesyntax}
\tsmacro{\typesetverb}[obj-type]{verbatim text}
\tsmacro{\tsverb}[obj-type]{verbatim text}
\end{codesyntax}
 Typesets \tsobj[marg]{verbatim text} as is (verbatim...). \tsobj[oarg]{obj-type} defines the used format. The difference with \tsverb{\tsobj[verb]{something}} is that \tsmeta{verbatim text} can contain commas (which, otherwise, would be interpreted as a list separator in \tsobj{\tsobj}.
\end{codedescribe}
\begin{tsremark}
This is meant for short expressions, and not multi-line, complex code (one is better of, then, using \ref{codelist}).  \tsobj[meta]{verbatim text} must be balanced ! otherwise, some low level \TeX\ errors may pop out.
\end{tsremark}


\begin{codedescribe}[code]{\typesetmarginnote,\tsmarginnote}
\begin{codesyntax}
\tsmacro{\typesetmarginnote}{note}
\tsmacro{\tsmarginnote}{note}
\end{codesyntax}
Typesets a small note at the margin.
\end{codedescribe}

\begin{codedescribe}[env,update=2025-04-21]{tsremark,tsremark*}
\begin{codesyntax}
\tsmacro{\begin{tsremark}}[NB]{}
\tsmacro{\end{tsremark}}{}
\end{codesyntax}
 The environment body will be typeset as a text note. \tsobj[oarg]{NB} (defaults to Note:) is the note begin (in boldface). The non-star version doesn't finishes a paragraph (\TeX\ stays in horizontal mode), whilst the (new) star version does and introduces a vertical space at the end. For instance:
 \begin{codestore}[demo.remark]
 Sample text. Sample test.
  \begin{tsremark}[N.B.]
    This is an example.
  \end{tsremark}
 \end{codestore}
 \tsdemo{demo.remark}
\end{codedescribe}

\subsection{Auxiliary Commands and Environment}
In case the Document Class being used redefines the \tsobj[code]{\maketitle} command and/or \tsobj[env]{abstract} environment, alternatives are provided (based on the article class).

\begin{codedescribe}[code]{\typesettitle,\tstitle}
\begin{codesyntax}
\tsmacro{\typesettitle}{title-keys}
\tsmacro{\tstitle}{title-keys}
\end{codesyntax}
This is based on the \tsobj[code]{\maketitle} from the \tsobj[pkg]{article} class. The \tsobj[marg]{title-keys} are:
\end{codedescribe}

\begin{describelist}{key}
\describe{title}{The used title.}
\describe{author}{Author's name. It's possible to use the \tsobj[code]{\footnote} command in it.}
\describe{date}{Title's date.}
\end{describelist}

\begin{codedescribe}[env]{tsabstract}
\begin{codesyntax}
\tsmacro{\begin{tsabstract}}{}
\ldots
\tsmacro{\end{tsabstract}}{}
\end{codesyntax}
This is the \tsobj[env]{abstract} environment from the \tsobj[pkg]{article} class.
\end{codedescribe}

\begin{codedescribe}[code,new=2023/05/16]{\typesetdate,\tsdate}
\begin{codesyntax}
\tsmacro{\typesetdate}{}
\tsmacro{\tsdate}{}
\end{codesyntax}
This provides the current date (Month Year, format).
\end{codedescribe}

\end{document} 